\documentclass[12pt]{article}
\usepackage{amsmath,amssymb}

\newcommand{\C}{{\mathbb C}}
\newcommand{\R}{{\mathbb R}}
\DeclareMathOperator{\range}{range}
\DeclareMathOperator{\Null}{null}
\DeclareMathOperator{\Real}{Real}

\title{AMCS 390 Fall 2017 Homework 2}

\begin{document}
\date{}
\maketitle

Come prepared to present your solutions on Monday, August 28th.
\begin{enumerate}

    \item   A circulant matrix is any $m\times m$ matrix that can be written as
            $A=\sum_{k=0}^{m-1} a_{k+1}C^k$, where $C$ is the cyclic shift matrix:

            $$C = \begin{bmatrix} 0 & 1 & 0 & \cdots & 0 \\ \vdots & 0 & 1 & & \vdots \\
              0 & & \ddots & \ddots & 0  \\ 0 & & \cdots & 0 & 1\\ 1 & 0 & & \cdots & 0\end{bmatrix}$$

            Show that the eigenvectors of any such matrix are discrete Fourier modes
            with entries $v_j = e^{ij\xi}$.  What are the appropriate values of $\xi$?

    \item   Determine the numerical dispersion relation for the heat equation semi-discretized with
            second-order centered differences.  How does this differ from the true dispersion
            relation?

\end{enumerate}

\end{document}
