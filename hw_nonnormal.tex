\documentclass[12pt]{article}
\usepackage{amsmath,amssymb}

\newcommand{\C}{{\mathbb C}}
\newcommand{\R}{{\mathbb R}}
\DeclareMathOperator{\range}{range}
\DeclareMathOperator{\Null}{null}
\DeclareMathOperator{\Real}{Real}

\title{AMCS 390 Fall 2017 Homework}

\begin{document}
\date{}
\maketitle

Come prepared to present your solutions on Thursday, November 30th.

\begin{enumerate}

    \item Consider the advection IBVP on the interval $0\le x \le 1$:
        \begin{align*}
            u_t & = u_x \\
            u(x=1,t) & = 0 \\
            u(x,t=0) & = u_0(x).
        \end{align*}

        (a) Semi-discretize in space with first-order upwinding, to obtain a linear
            system of ODEs $u'(t) = Au(t)$.  What is the spectrum of $A$?  What do
            its $\epsilon$-pseudospectra look like?  You can try using random
            perturbations or a pseudospectra package to answer the latter question.
        
        (b) Discretize in time with the forward Euler method.  What does traditional
            method-of-lines stability analysis say about the stable time step size?
            In other words, what is the largest step size such that $h\lambda$ is
            in the region of absolute stability?

        (c) Implement this full discretization.  What happens in practice if
            you take the step size indicated in part (b)?  Be sure to consider what
            the numerical solution does over both short and long times.

        (d) What does the CFL condition say?

        (e) Can you reconcile all of these answers based on the pseudospectra?

\end{enumerate}

\end{document}
