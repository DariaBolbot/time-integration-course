\documentclass[12pt]{article}
\usepackage{amsmath,amssymb}

\newcommand{\C}{{\mathbb C}}
\newcommand{\R}{{\mathbb R}}
\DeclareMathOperator{\range}{range}
\DeclareMathOperator{\Null}{null}
\DeclareMathOperator{\Real}{Real}

\title{AMCS 390 Fall 2017 Homework 5}

\begin{document}
\date{}
\maketitle

Come prepared to present your solutions on Thursday, November 16th.

\begin{enumerate}

    \item Consider the two-point BVP
        \begin{align*}
            \epsilon u'' & = a u' + b(t)u + q(t) \\
            u(0) & = b_1 \\
            u(1) & = b_2,
        \end{align*}
        where $a\ne0$ is a constant and $b, q$ are continuous functions, all
        ${\mathcal O}(1)$.

        (a) Write the ODE in first-order form for the vairables $y_1=u$ and
            $y_2 = \epsilon u' - au$.
        
        (b) Letting $\epsilon \to 0$, show that the limit system is an index-1 DAE.

        (c) Show that only one of the boundary conditions is needed to
            determine the solution of the DAE.  Which one?  Why?

    \item Consider the DAE
        \begin{align*}
            y_1' & = y_3 y_2' - y_2 y_3' \\
            0 & = y_2 \\
            0 & = y_3.
        \end{align*}
    
        (a) Show that the DAE has index 1.

        (b) Show that if we add to the right-hand side the (small) perturbation

                $$\delta(t) = (0, \epsilon\sin(\omega t), \epsilon\cos(\omega t))^T$$

            then in the perturbed solution $y_1'(t) = \epsilon^2 \omega$, which is unbounded
            as $\omega \to \infty$.  DAEs are unstable in the sense that small perturbations
            to the algebraic equations can induce large perturbations in the solution.
\end{enumerate}

\end{document}
